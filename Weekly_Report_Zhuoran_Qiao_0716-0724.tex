% !Mode:: "TeX:UTF-8"

\documentclass[11pt, a4paper]{article}
%\usepackage{xltxtra,fontspec,xunicode}
\usepackage{amsmath}
\usepackage{amssymb}
\usepackage{breqn}
\usepackage{autobreak}
\usepackage{braket,mleftright}
\usepackage{amsfonts}
\usepackage[section]{placeins}
\usepackage{float}
\usepackage{siunitx}

\usepackage{indentfirst}
\usepackage{caption}

\usepackage{geometry}
\usepackage{graphicx}

\geometry{top=1in, bottom=1in, left=1in, right=1in}
\linespread{1.5}

\DeclareMathOperator*{\argmax}{argmax}
\DeclareMathOperator*{\argmin}{argmin}

\newcommand{\degc}{$\,^\circ$C}

\begin{document}

\title{Weekly Report}
\author{Zhuoran Qiao}
\date{\today}

\maketitle

\section{Introduction}
\paragraph{a.} Derived formalisms for transition path statistics based on Transition Path Theory and continuous-time random walk (CTRW) model;
\paragraph{b.} Suggested an simple method to analytically predict transition path time (TPT) distribution on arbitrary 1-d free energy landscapes (FELs);
\paragraph{c.} Calculated probability distribution of backward jumping within transition paths ensemble for 1-d toy landscapes.

\section{Progress}
\subsection{Transition path statistics of CTRW model}

\paragraph{}Our goal is to figure out a general approach to calculate transtion path statistics (average transit time and transit time distribution)
 for free energy landscape with discrete states. The Markov rate matrix $\bold{W}$ element between state $\sigma_i$ and state $\sigma_j$ is given by

 \begin{equation}
   \bold{W}_{ij}=\langle \sigma_i | \bold{W} | \sigma_j \rangle = k_0 \exp(-\frac{F_j-F_i}{2})
 \end{equation}

Based on transition path theory, the probability current of reactive trajectories between $\sigma_i$ and state $\sigma_j$ $(i \neq j)$ is $v_{ij}$

\begin{equation}
  v_{ij}=\pi_i q_i^- q_j^+ \langle \sigma_i | \bold{W} | \sigma_j \rangle
\end{equation}

Where $q^+$ and $q^-$ are repectively forward and backward committors.

We can prove that transition paths of such a free energy landscape are equivalent to paths generated by a CTRW process with following parameters;

Jump matrix $\bold{Q}$:

\begin{equation}
  \bold{Q}_{ij}=\langle \sigma_i | \bold{Q} | \sigma_j \rangle = \frac{\bold{W}_{ij} q_j^+}{\sum_k \bold{W}_{ik} q_k^+}
\end{equation}

Waiting time $\tau$ distribution at state $\sigma_i$:

\begin{equation}
  P(\tau|\sigma_i)=\frac{1}{\theta_i}\exp(-\tau/\theta_i)
\end{equation}

As transition path transit time (TPT) $t$ is equal to first-passage time in that CTRW model, our problem is reduced to a mathematical problem to solving first hitting time at sink $B$.

We follow notations in \cite{Manhart2015}, and denote hitting counts vector $\vec{H}=[H_A, H_1, H_2,..., H_i, ...H_B ]$ as a functional of path $\varphi$,
 where $H_i$ is the counts that state $\sigma_i$ is hitted within path $\varphi$;

Rewrite TPT distribution function $P(t)$ in following expansions, where $l$ is the length of $\varphi$

\begin{equation}
  \begin{split}
    P(t)&=\sum_{\varphi} P(t|\varphi)P(\varphi)\\
        &=\sum_{\varphi} P(t|\vec{H}[\varphi])P(\varphi)\\
        &=\sum_{\vec{H}} P(t|\vec{H})P(\vec{H})\\
        &=\sum_{\vec{H},l} P(t|\vec{H})P(\vec{H}|l)P(l)
  \end{split}
\end{equation}

For 1-d FELs, we proved that the distribution of $H_i$ is exactly given by

\begin{equation}
  H_i \sim Geom(\bold{Q}_{i,i+1} g_{i+1}^+)
\end{equation}

Where the 'secondary committor' $g_i$ is given by another Dirichlet problem:

\begin{equation}
  \begin{split}
    \sum_j \bold{Q}_{kj} g_j =0, \ k \notin (i\cup B)\\
    g_B =1, \ g_i =0.
  \end{split}
\end{equation}

However $H_i$ and $H_j$ are correlated; $P(H_i|H_j)\neq P(H_i)$. Only when we considering highly coarse-grained states
 $P(\vec{H})=\prod_i P(H_i)$ is good approximation.

Alternatively we consider the case that most paths are sufficiently long, then

\begin{equation}
  P(l)=(\bold{Q}^n)_{AB}-\sum_{k=1}^{l-1} P(k) (\bold{Q}^{n-k})_{AB} \sim \exp(-\alpha l/ \bar{l}),\ \bar{l}=-1/log(q)
\end{equation}

Where q is the largest eigenvalue of $\bold{Q}$.

Applying central limit theorem, $\vec{H}$ only depends on $\vec{p}$ that
 \[\vec{p}=[P(A|TP), P(1|TP), P(2|TP), ...P(i|TP), P(B|TP)]\]
\begin{equation}
  P(\vec{H}|l)\sim Multinomial(\vec{p})
\end{equation}

If $\| \vec{H} \| \gg l$, $P(\vec{H}|l)\rightarrow \italic{N}(l\vec{p},\ l [\bold{Diag}(\vec{p})-\vec{p}\,\vec{p}^T])$, $E(\vec{H}|l)=l\vec{p}$.

$P(t|\vec{H})$ as a convolution of $l$ exponential random variables is analytically solvable:

\begin{equation}
  P(t|\vec{H})=1-(\prod_{j=A}^{B} \theta_{j}^{-H_j})\sum_{k=A}^{B}\sum_{m=1}^{H_k} \frac{\psi_{k,\,m}(-1/\theta_k)t^{H_k-1}\exp(-t/\theta_k)}{(H_k-m)! (m-1)}
\end{equation}

Where
\begin{equation}
  \phi_{k,\,m}(u)=-\frac{\partial^{m-1}}{\partial u^{m-1}}\{ \prod_{j=A,\, j\neq k}^{B} (\frac{1}{\theta_j}+1)^{-H_j}\}
\end{equation}

 implies...

\subsection{`Reverse diffusion map' method to predict TPT distribution on 1-d FELs}


\subsection{Finite time displacement distribution}

\begin{figure}[htp]
  \noindent\makebox[\textwidth]{\includegraphics[width=\textwidth]{TP_distribution_cp.png}}
  \caption{Comparision of $p(TP|x)$ for harmonic, flattened harmonic, 2 barriers and high frequency barriers free energy models.}
  \label{fig:tpx_cp}
\end{figure}

\paragraph{}To preliminarily test this assumption, we constructed a model free energy landscape with 50 intermediate barriers('Highfreq'), as well as a 'flatted' harmonic model in which $\Delta F= 1.34 kT$ (Figure \ref{fig:FELs_2}).
I repeated $p(TP|x)$ calulation for these models (Figure \ref{fig:tpx_cp}), and the result showed that the distribution of $p(TP|x)$ of high frequency model strongly resembled the flatted harmonic model near the barrier region,
 indicating well separation of time scale for the high frequency free energy landscape model.

\textbf{That result is probably important for giving explanation to the fact that protein folding transition-path statistics can be described
 by one-dimentional coordinate, but the empirical $\Delta F$ is much lower than real barrier height.}

\section{Future plan}

\paragraph{a.} KMC sampling for transition paths on high frequency model and flattened harmonic model is required to further test the assumption in $2.4$, which can be finished on next week; larger scale simulation is also needed to refine transit time distribution data.
\paragraph{b.} Deciding future proposal for synonymous codon usage project.

%In order to examine when the one-dimensional coordinate projection could be recognized as effective reaction coordinate,
%we then examine the probability distribution of committors for transition path trajectories $p(q|TP)$, for which a single peak of probability $p(q|TP)$ have been ultilized as a indicator for 'good' reaction coordinates.
% Firstly, We found that for harmonic toy model, the shape of $p(q|TP)$ is very sensitive to the definition of source/sink region. For illustration, $p(q|TP)$ for two different selection
%  of source/sink regions was compared: in the first case, only two free energy minimum was indentified as source or sink\textbf{(S1)}; in the second case, only the barrier top was
%  defined as the transition path, while other two region of the free energy landscape was calssified as source/sink\textbf{(S2)}.

%\cite{Jacobs2018,Chaudhury2010,Vanden-Eijnden2010,Metzner,Krivov}

\small
\bibliographystyle{plain}
\bibliography{ref}

\end{document}
